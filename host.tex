\chapter{Host- und Betriebssystem-Security}
Ziele:
\begin{itemize}
    \item Schutz der Daten
    \item Schutz der laufenden Prozesse
    \item Schutz des Betriebssystems
\end{itemize}

Best-practice:
\begin{itemize}
    \item Saubere Grundinstallation
    \item Arbeiten als eingeschränkter User
    \item \qq{Nur das nötigste} installieren/aktivieren
    \item OS-Hardening (Siehe: Center for Internet Security, Bastille Linux, Windows Defender)
    \item Nutzung von Sicherheitssoftware:
        \begin{itemize}
            \item Bessere bzw. erweiterte Zugangskontrolle (z.B. ACLs)
            \item Verschlüsselung (ssh statt telnet, TLS)
            \item OS-Erweiterungen (SELinux, AppArmor)
            \item Virenscanner, Spyware-Detection, Firewalls
                (erhöht aber natürlich auch die \qq{Attack Surface})
        \end{itemize}
    \item Netzwerk Auditing:
        \begin{itemize}
            \item MS Security Baseline Analyzer
            \item Nessus (bzw. OpenVAS)
            \item Security Local Audit Daemon
        \end{itemize}
\end{itemize}

\section{Windows Security Features}
\begin{itemize}
    \item Automatische Updates
    \item (Aktueller) Virenscanner
    \item Integrierte Firewall \& Defender Lösungen
    \item Benutzer ohne Admin-Rechte
    \item User Account Control (UAC)
    \item Integrity Levels bei Benutzer Prozessen
    \item Windows Defender
    \item Verbesserungen in IE
    \item ASLR
    \item Kernel Patch Protection
    \item Data Execution Prevention
    \item Bitlocker Festplattenverschlüsselung
\end{itemize}

\section{Linux Security Features}
\begin{itemize}
    \item No open ports
    \item Salted SHA-512 PW hashes
    \item SYN cookies
    \item Filesystem capabilities
    \item iptables
    \item Einschränkung der Syscalls durch PR\_SET\_SECCOMP
    \item Mandatory Access Control mit AppArmor, SELinux oder SMACK
    \item Filesystemverschlüsselung mit eCryptFS
    \item Userspace hardening:
        \begin{itemize}
            \item Stack Protector
            \item Heap Protector
            \item Pointer Obfuscation
            \item ASLR (mit Position Independent Executable)
            \item -D\_FORTIFY\_SOURCE für Funktionen ohne Stack-Buffer-Overflow Probleme
            \item NX-Bit
        \end{itemize}
    \item Kernelspace hardening:
        \begin{itemize}
            \item 0-address protection
            \item /dev/kmem disabled
            \item Block module loading
            \item Read-only data sections
            \item Stack protector
            \item Module RO/NX
            \item Limiting kernel information
            \item Filter syscalls
        \end{itemize}
\end{itemize}

\subsection{SELinux}
SELinux implementiert eine \qq{mandatory access control} Architektur für den Linux Kernel,
 entwickelt von der NSA und seit 2003 vom Mainstream-Kernel.

\subsection{AppArmor}
Implementiert eine MAC als Kernel-Modul über die \qq{Linux Security Modul}(LSM) Schnittstelle, ähnlich zu SELinux

\section{MacOS Security Features}
Userspace:
\begin{itemize}
    \item Stack Guard
    \item Full ASLR
    \item Full NX
    \item Fortify Source
    \item Gatekeeper
    \item Full disk encryption
    \item Sandboxing
    \item Privacy Controls
\end{itemize}

Kernel Space:
\begin{itemize}
    \item Kernel ASLR
    \item Separate address spaces
    \item User Approved Kernel Extension Loading
    \item Firmware validation
    \item System Integrity Protection
\end{itemize}

\section{iOS Security}
\begin{itemize}
    \item Hardware Security Modul schützt Schlüssel
    \item Secure Boot Prozess
    \item Isolation und App Sandboxing
    \item Code Signing und Whitelisting
    \item Hardened Webkit JiT Mapping (JiT Speicherbereich zweimal im Speicher, einmal XR und einmal W-only)
    \item Enge Integration von Crypto in Hardware
    \item Master-Key an Hardware gebunden
    \item Auch sicher bei manipuliertem Application Host
\end{itemize}

\section{Referenzmonitore}
\subsection{Begriffe}
\subsubsection{Referenzmonitor}
Abstraktes Sicherheitskonzept, welches alle Zugriffe von Subjekten auf Objekte kontrolliert.

\subsubsection{Security Kernel}
Implementierung des Referenzmonitors

\subsubsection{Trusted Computing Base (TCB)}
Beinhaltet Security Kernel plus weitere Sicherheitsmaßnahmen

\subsection{Anforderungen}
\begin{itemize}
    \item Manipulationssicher
    \item Muss immer aufgerufen werden
    \item Klein genug für Analyse und Tests zur Verifikation dr Korrektheit
\end{itemize}

\section{Virtualisierung}
Vorteile:
\begin{itemize}
    \item Reduktion von Hardware-Maschinen
    \item Ausführung verschiedener Betriebssysteme
    \item Fehlersuche, Testumgebung
    \item Sandboxign
    \item Isolation von Systemen und Anwendungen
    \item Bei Kompromittierung leichter Wiederanlauf
    \item Isolation von Fehlern
    \item Sandboxing
    \item Intrusion-Detection
    \item Leichte Integration von Trusted Computin
    \item Remote-Management
\end{itemize}

Probleme:
\begin{itemize}
    \item Führt oftmals zu \qq{Wildwuchs} von Systemen
    \item Z.t. nicht aktuell wenn länger deaktiviert
    \item Auch Virtualisierungssysteme sind nicht fehlerfrei
    \item Kann auch von Rootkits ausgenutzt werden
\end{itemize}

\subsection{Formen der Virtualisierung}
\subsubsection{Vollvirtualisierung}
Gast-OS läuft in Virtual Machine wie auf \qq{echter} Hardware.

\subsubsection{Paravirtualisierung}
Gast-OS erfordert Anpassungen an Wirtssystem, weiß damit von der Virtualisierung.


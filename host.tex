\chapter{Host- und Betriebssystem-Security}
Ziele:
\begin{itemize}
    \item Schutz der Daten
    \item Schutz der laufenden Prozesse
    \item Schutz des Betriebssystems
\end{itemize}

Best-practice:
\begin{itemize}
    \item Saubere Grundinstallation
    \item Arbeiten als eingeschränkter User
    \item \qq{Nur das nötigste} installieren/aktivieren
    \item OS-Hardening (Siehe: Center for Internet Security, Bastille Linux, Windows Defender)
    \item Nutzung von Sicherheitssoftware:
        \begin{itemize}
            \item Bessere bzw. erweiterte Zugangskontrolle (z.B. ACLs)
            \item Verschlüsselung (ssh statt telnet, TLS)
            \item OS-Erweiterungen (SELinux, AppArmor)
            \item Virenscanner, Spyware-Detection, Firewalls
                (erhöht aber natürlich auch die \qq{Attack Surface})
        \end{itemize}
    \item Netzwerk Auditing:
        \begin{itemize}
            \item MS Security Baseline Analyzer
            \item Nessus (bzw. OpenVAS)
            \item Security Local Audit Daemon
        \end{itemize}
\end{itemize}

\section{Windows Security Features}
\begin{itemize}
    \item Automatische Updates
    \item (Aktueller) Virenscanner
    \item Integrierte Firewall \& Defender Lösungen
    \item Benutzer ohne Admin-Rechte
    \item User Account Control (UAC)
    \item Integrity Levels bei Benutzer Prozessen
    \item Windows Defender
    \item Verbesserungen in IE
    \item ASLR
    \item Kernel Patch Protection
    \item Data Execution Prevention
    \item Bitlocker Festplattenverschlüsselung
\end{itemize}

\section{Linux Security Features}

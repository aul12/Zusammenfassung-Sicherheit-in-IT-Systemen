\chapter{Netzwerk-Security}
\section{Vorgehen bei Angriffen auf Netzwerke}
\begin{itemize}
    \item Informationen beschaffen und Netzwerk analyisieren:
        \begin{itemize}
            \item Wie ist das Netz strukturiert? Verbindungen? Hosts? Segmente?
            \item Welche Basis-Dienste laufen im Netz? DNS? DHCP? Mail? SSH? Webserver? SMB-Server? VOIP?
            \item Welche Infos bekommt man durch Sniffen raus
        \end{itemize}
    \item Erkennen von Schwachstellen und Durchführen von Angriffen
        \begin{itemize}
            \item Suche nach bekannten Schwachstellen
            \item Falsche Pakete schicken, Protokolle austricksen, Datenverkehr umleiten
            \item Einzelne Hosts und Ports/Dienste genauer untersuchen
        \end{itemize}
    \item Dauerhafte Kontrollübernahme: Installation von rootkits, backdoors,...
    \item Weiter Ausbreiten
        \begin{itemize}
            \item Suche nach Credentials, die Zugriff auf andere Systeme gestattten
            \item Suche nach weiteren Systemen, die vim übernommenen Host erreichbar sind
        \end{itemize}
\end{itemize}

\section{Informationsbeschaffung}
\subsection{DNS}
Einfache Informationbeschaffung: \qq{State of Authority} (SOA) Record mit Dig
\begin{lstlisting}
    dig uni-ulm.de SOA
\end{lstlisting}
oder Zonedumps (alle Einträge zu einer Domain):
\begin{lstlisting}
    dig uni-ulm.de axfr
\end{lstlisting}

\vspace{.3cm}

Information über Besitzer einer Domain:
\begin{lstlisting}
    whois uni-ulm.de
\end{lstlisting}

\vspace{.3cm}

Information über den Netzwerkaufbau
\begin{lstlisting}
    traceroute -I uni-ulm.de
\end{lstlisting}

\vspace{.3cm}

Mit SMTP-Mailserver reden, liefert oftmals Information über Software:
\begin{lstlisting}
    telnet mail.uni-ulm.de 25
\end{lstlisting}

Sniffer, z.B. tcpdump, ethereal protokollieren Netzwerk auf gemeinsam genutzten Netzwerksegment.

\vspace{.3cm}

\lstinline{smbclient} (bzw. \lstinline{net} unter Windows) liefern Informationen zu SMB-Shares.

\vspace{.3cm}

TCP \lstinline{connect()} oder SYN scanning mit NMAP möglich, dadurch offene Ports und z.T. auch OS bekannt.

\section{Angriffsvektoren}
TCP/IP wurde nicht als sicheres Protokoll entworfen, mögliche Angriffe sind:
\begin{itemize}
    \item IP-Spoofing
    \item TCP Hijacking
    \item Cache Poisoning (DNS, ARP)
    \item Routing Angriffe
\end{itemize}

\subsection{Routing}
Ziele:
\begin{itemize}
    \item Traffic Redirection
    \item Blackhole Routing
\end{itemize}

Leicht Angreifbar:
\begin{itemize}
    \item Im eigenen Subnetz:
        \begin{itemize}
            \item ARP Spoofing
            \item DNS Spoofing/Cache Poisoning
            \item ICMP Redirects/Router Discovery Protocol
            \item Fake DHCP replies
        \end{itemize}
    \item Interior Routing:
        \begin{itemize}
            \item RIP: Unsicher
            \item OSPF: Einfache Authentifizierung (Klartextpasswort/MD5-Hash)
        \end{itemize}
    \item Exterior Routing:
        \begin{itemize}
            \item BGP4: Manuelle Filter bestimmen ob Information angenommen wird
        \end{itemize}
\end{itemize}
    
\subsection{DNS}
Angreifbar:
\begin{itemize}
    \item \lstinline{/etc/hosts}-File angreifbar
    \item DNS-Hijacking:
        \begin{itemize}
            \item Stehlen der Domäne, z.B. nach Ablaufen
            \item Antwort für nicht existente Domänen (z.B. googel.com)
        \end{itemize}
    \item DNS Cache Poisoning:
        \begin{itemize}
            \item Angreifer sendet Antwortpakete anstelle des DNS Server
            \item Aber: Angreifer muss die Query-ID kennen
        \end{itemize} 
\end{itemize}

\subsubsection{Einfache DNS Cache Poisoning Attacke}
Herausforderungen für den Angreifer:
\begin{itemize}
    \item Name darf noch nicht im Cache sein
    \item Korrektes Raten der Query ID
    \item Angreifer muss schneller sein als echter Nameserver
\end{itemize}

Gegenmaßnahmen: Randomisierte Query IDs

\vspace{.3cm}

Gegengegenmaßnahme: Anfragen fluten, in circa 10 Sekunden passt dann irgendwann die QID,
Direkt \lstinline{.com} resolven.

\vspace{.3cm}

Gegengegengegenmaßnahmen: Randomisierung des Source-Ports.

\vspace{.3cm}

Langfristig: DNSSEC

\section{Denial of Service}
Kann sich gegen quasi jede Komponente richten:
\begin{itemize}
    \item Hardware (z.B. Stromausfall)
    \item OS: z.B. SYN-flood; ping-of-death
    \item Netzwerk: Overflow, Amplifier, Router-DoS, DNS-Server
    \item Benutzer: Spam
\end{itemize}

Mit mehreren Angreiferen: Distributed DoS (DDOS). Auch \qq{einfach Einzukaufen}.

\section{Netzwerk Security}
Komponenten der Netzwerk Security sind:
\begin{itemize}
    \item Firewall
    \item Intrusion Detection/Reaction Systeme
    \item Schutz des Netzwerkzugangs und der aktiven Netzwerkkomponenten:
        \begin{itemize}
            \item Management von routern, switches etc.
            \item Physikalischer Zugang zu Ports, Schränken etc.
            \item Überwachung, z.B. mit arpwatch
        \end{itemize}
    \item Content-Security
        \begin{itemize}
            \item Automatischer Virenfilter für Emails, am Web-Proxy, etc.
            \item Automatische Verschlüsselung
        \end{itemize}
    \item Wireless Security, Mobilgeräte etc.
\end{itemize}

\section{Firewalls}
Eine Firewall ist eine Netzwerkkomponente, die mindestens zwei Netzwerksegmente voneinander trennt und dabei den Netzwerkverkehr gemäß einer Sicherheitsrichtlinie kontrolliert und gegebenfalls unterbindet.

\vspace{.3cm}

Filterung der Datenströme:
\begin{itemize}
    \item Virus-/Java-/ActiveX Filter für Emails, HTTP,...
        \begin{itemize}
            \item Zentrales Filtering an einer Stelle mit aktuellen Signaturen
            \item Umgehung des Filters durch ungewöhnliche Kompressionsverfahren etc. möglich
        \end{itemize}
    \item Inhaltsfilter: Zieladressen, Schlüsselwort, etc.
    \item Anti-Spoofing: Entfernt Pakete mit gefälschten Adressen oder Broadcasr Adressen
\end{itemize}

\vspace{.3cm}

Weitere Funktionen:
\begin{itemize}
    \item Logging der Internet-Aktivitäten
    \item Authentifizierung und Rechtevergabe an einzelne Mitarbeiter
    \item Bessere datenschutzrechtliche Kontrolle
    \item NAT
\end{itemize}

\vspace{.3cm}

Zu einer Firewall gehören:
\begin{itemize}
    \item Sicherheitsrichtlinien
    \item Router
    \item Paketfilter
    \item Circuit Level Gateways
    \item Application Level Gateways
    \item Authentisierungsmaßnahmen
    \item Mechanismen für Logging
\end{itemize}

\subsection{Security Policy}

\subsubsection{Default-Deny}
Zugriff ist generell verboten, sofern er nicht explizit erlaubt ist.

Folge: nur wirklich benötigte Dienste werden freigegeben.

Nachteil: Nutzerakzeptanz. Alle neuen Dienste müssen händisch freigeschalten werden.

\subsubsection{Default-Allow}
Zugriff ist generell erlaubt, sofern nicht explizit verboten.

Folge: Gefährliche Dienste werden gesperrt.

Nachteil: Der Administrator muss alle gefährlichen Dienste kennen.

\subsection{Filterung des Netzwerkverkehrs}
\subsubsection{Network-level Filtering}
\paragraph{Layer 3}
Filterung auf Basis von IP Paketen (Sender/Empfänger)

\paragraph{Layer 4}
Zusätzliche Filterung auf Basis von UDP/TCP Ports und Richtung des Verbindungsaufbaus

\subsubsection{Application-level Filtering}
Betrachtet auch die jeweiligen Anwendungsprotokolle, z.B. nur bestimmte HTTP Methoden oder URL, nur bestimmte Mailadressen.

\subsection{Firewall Architekturen}
\subsubsection{Packet Filtering Router}
Router arbeitet direkt als Firewall, einfachst mögliche Architektur.

\subsubsection{Screened Host}
Es existiert eine De-Militarisierte Zone (DMZ), in diese sind Screening Router und Application Gateway. Jeglicher Traffic wird vom Application Gateway gefiltert.

\subsubsection{Dual-Homed Host}
Der Server (DHH) ist mit beiden Netzwerken (intern und extern) Verbunden, es findet aber kein Routing dazwischen statt.

\subsubsection{Screened Subnet}
Es gibt einen Exterior-Router und einen Interior-Router in der DMZ, beide kommunizieren nur mit dem Application Gateway und ihrem jeweiligen Bereich.

\subsubsection{Mehrstufiges Firewall Konzept}
Kombination von oben vorgestellten Konzepten.

\subsection{Nicht gelöste Probleme}
\begin{itemize}
    \item Eingeschränkter Zugriff für legitime Benutzer
    \item Kein Schutz vor Insider Attacken
    \item Keine Kontrolle von Hintertüren (z.B. Modems, WLANS)
    \item Akzeptanzprobleme
\end{itemize}

\section{Intrusion Detection Systeme}
Ziel: Erkennung von Angriffen auf ein System.

\vspace{.3cm}

Unterscheidung in der Funktionsweise:
\begin{itemize}
    \item Host-basiert (HIDS):
        \begin{itemize}
            \item Wird auf Zielsystem installiert
            \item Wertet Informationen des OS aus
            \item Erkennt Angriffe auf einzelnes System
        \end{itemize}
    \item Netwerk-basiert (NIDS):
        \begin{itemize}
            \item Wird an ein oder mehreren Stellen im Netzwerk verbunden
            \item Wertet Verkehr aus
            \item Erkennt Angriffe auf das Netzwerk
        \end{itemize} 
    \item Hybrideansätze
\end{itemize}

\vspace{.3cm}

Unterscheidung im Vorgehen:
\begin{itemize}
    \item Signatur basiert
    \item Anomalie basiert
    \item Spezifikation basiert
\end{itemize}

\vspace{.3cm}

Bei Erkennung sollte eine Reaktion erfolgen:
\begin{itemize}
    \item Verbindung unterbrechen
    \item Firewall umkonfigurieren
    \item Pakete modifizieren
\end{itemize}

\chapter{Embedded \& Hardware Security}
Allgemein: Viele Security Mechanismen können von tieferen Schichten kompromittiert werden.
Hardware ist tiefste Ebene.

\section{Privilege Levels}
Mehrere Schutzringen in 80x86 Prozessoren:
\begin{enumerate}
    \setcounter{enumi}{-2}
    \item Virtual Machine Monitor
    \item Kernel
    \item OS
    \item I/O-Treiber
    \item Userspace
\end{enumerate}

Datenzugriff nur auf eigener oder auf höherer Ebene, Funktionszugriff nur auf eigener Ebene.

\vspace{.3cm}

Jedes Speichersegment enthält ein \qq{Descriptor Privilege Level} (DPL)

Prozesse haben ein \qq{Requested Privilege Level} (RPL) und ein \qq{Current Privilege Level} (CPL).

Falls DPL $< \max($CPL$,$ RPL$)$ wird eine Schutzverletzung ausgelöst.

Wechsel in höheren Ring durch Gates (Systemaufrufe).


\section{Seitenkanalangriffe}
Mögliche Seitenkanäle:
\begin{itemize}
    \item Stromaufnahme
    \item EM-Strahlung
    \item Akustisch (z.B. Lüfter, Drucker, Position auf Touchscreen durch akustische Trilateriation)
\end{itemize}

\subsection{Rowhammer und RAMBleed}
Durch häufiges Lesen im Speicher werden Bitflips produziert.

\subsection{Meltdown und Spectre}
Speculative Execution wird ausgenutzt indem Ladezeiten für Daten gemessen wird, 
wenn bereits geladen wurde dann in Cache und schneller.

\section{Trusted Plattform Module (TPM)}
Spezieller Hardwarebaustein, der Kryptooperationen und Schlüsselmanagement unterstützt.

\section{Smartcards}
Ermöglichen:
\begin{itemize}
    \item Sichere Identifikation
    \item Sicherer Datenspeicher
    \item Sichere Ausführungsumgebung
\end{itemize}

Bestandteile:
\begin{itemize}
    \item CPU
    \item ROM
    \item RAM
\end{itemize}

Schutzfunktionen:
\begin{itemize}
    \item Security by obscurity auf Hardware-Ebene
    \item Interne Busse
    \item ROM in tiefen Siliziumschichten
    \item Ionenimplantierte ROM-Code (verhindert Auslesen mit Lichtmikroskop)
    \item Abschirmung des EEPROM um Abstrahlung zu verhindern
    \item Physikalische Zerstörung bei Manipulationsversuch
    \item Messung der Umgebung
    \item Trap-Door Funktionen (irreversibler Speicher Schreibschutz)n
    \item Stromverbrauchsangleichung
\end{itemize}

\subsection{Seitenkanalangriffe}
\begin{itemize}
    \item Laufzeitangriffe
    \item Cacheangriffe
    \item Power-Attacken
    \item Abstrahlangriffe
\end{itemize}

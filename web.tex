\chapter{Web Security}
Mögliche Angriffe:
\begin{itemize}
    \item Angriffe auf Server/Web-Anwendungen:
        \begin{itemize}
            \item URL-Manipulation
            \item Fehler in Skripten
            \item Cross-Site-Scripting (XSS)
            \item SQL-Injection
            \item Remote Code Injection
            \item Authentication Bypass
        \end{itemize}
    \item Angriffe auf den Browser: Engine Exploits (z.B. Renderer, Scripting, Java, Flash)
    \item Angriffe auf die Infrastruktur: Missbrauch von TCP/IP Mechanismen
\end{itemize}

\section{Angriffe}
\subsection{Injection}
Nie der Korrektheit von Argumenten vertrauen

\subsection{Access Control}
Zugriffsbeschränkung auf Test- / Admin- / Statusseiten fehlt häufig (z.B. phpinfo).

\subsubsection{Authentification Bypass}
\begin{itemize}
    \item Unzureichende Regeln für Passwort Komlexität
    \item Schwache Passwort-Rücksetzverfahren
    \item URL Manipulation (Presentation layer access control)
\end{itemize}

\subsection{Directory Traversal}
\qq{Ausbruch} aus htdocs Verzeichnis und Zugriff auf nicht-freigegebene Inhalte.

\subsection{Konfigurationsfehler}
Fehlerhafte Konfiguration führt zu Sicherheitsproblemen, es sind z.B. CGI Skripte über Foren einschleusbar.

\subsection{Cross Site Scripting}
Injection von clientseitigem Skriptcode möglich, dadurch manipulation der angezeigten Daten für andere User, oder Cookie stealing möglich.

\subsection{Injection Flaws}
z.B. SQL Injection, aber auch XPath, LDAP, Shell.

\subsection{Improper Error Handling}
Zu viele Informationen an User:
\begin{itemize}
    \item Detaillierte Versionsinfos
    \item Namen von Backend-Systemen
    \item Umgebungsvariablen
    \item Coredumps
    \item Aber auch z.B. \qq{not found} vs. \qq{access denied}
\end{itemize}

\subsection{Rogue Proxies}
Proxies können: Mithören, Inhalte verändern, Verbindungen blockieren.

\section{Best Practices}
\begin{itemize}
    \item Minimales Featureset aktivieren
    \item Versionsmeldungen abschalten
    \item Niemals Datenquellen von außen trauen
    \item Besondere Vorsicht bei Meta-Characters
    \item Alte Datein löschen
    \item Nicht im Live-System entwickeln
    \item Backupdateien entfernen (werden im Worst-Case als Textdatei ausgeliefert, also source Verfügbar)
    \item System aktualisieren
p   \item Fingerprinting erschweren (keine Fehlermeldung mit Software Bezeichnung/Versionsnummern)
\end{itemize}

\section{Secure Sockets Layer (SSL) / Transport Layer Security (TLS)}
Ziele: Authentisierung, Integrität und Vertraulichkeit:
\begin{itemize}
    \item Server-Authentifizierung mit Zertifikat
    \item Client Authentifizierung (Optional)
    \item Verschlüsselte Verbindung
\end{itemize}

Zwischen Anwendungsschicht und Transportschicht.

\subsection{SSL Handshake}
\begin{enumerate}
    \item Teilnehmerauthentisierung anhand von Zertifikation (Client optional)
    \item Vereinbarung der verwendeten kryptographischen Verfahren
    \item Vereinbarung des Sitzungsschlüssels
\end{enumerate}

In einer SSL-Session beliebig viele SSL-Connections.

\subsection{SSL Record Protocol}
\begin{itemize}
    \item Getrennt vom Handshake Protokoll
    \item Daten symmetrisch mit Session Key verschlüsselt
    \item MAC zur Integritätssicherung
    \item Getrennte Schlüssel für MAC und Verschlüsselung, jeweils in Sende- und
        Empfangsrichtung
\end{itemize}

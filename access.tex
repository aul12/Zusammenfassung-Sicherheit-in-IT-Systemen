\chapter{Zugriffskontrolle}
\section{Begriffe}
\subsection{Autorisierung}
Entscheidung, ob Subjekt $s$ Zugriff mit Operation $a$ auf Objekt $o$ hat.

\subsection{Zugriffskontolle}
Authentisierung + Autorisierung

\subsection{Subjekte}
Handelnde Einheit, die eine Operation durchführen möchte.

Z.B.: Person, Prozess, Netzwerkknoten.

\subsection{Objekt}
Einheiten auf welche zugegriffen wird.

Z.B.: Files, Verzeichnisse, Records in Datenbanken, Rechner

\section{Zugriffs-Operationen}
\subsection{Minimale Menge}
\begin{itemize}
    \item Observer
    \item Alter
\end{itemize}

\subsection{Bell-LaPadula}
\begin{itemize}
    \item Execute
    \item Append
    \item Read
    \item Write
\end{itemize}

\subsection{Weitere Operationen}
\begin{itemize}
    \item Grant/revoke
    \item Change Owner
\end{itemize}

\section{Organisation der Zugriffsrechte}
\subsection{Lampson's Access Control Matrix}
Für $S$ die Subjekte, $O$ die Objekte und $A$ die Operationen ist ein Element der Matrix
definiert als:
\begin{equation*}
    M_{so} \subseteq A\ \ \ s \in S, o \in O
\end{equation*}

\subsection{Access Control Lists (ACL)}
Speichert für jedes Objekt $o$, eine List von Tupeln $(s, a)$, die festlegt, dass Subjekt $s$ auf Objekt $o$ Operation $a$ durchführen darf.

\subsection{Capabilities}
Speichert für jedes Subjekt $s$ eine Liste von Tupeln $(a, o)$ die festlegt, dass Subjekt $s$ auf Objekt $o$ Operation $a$ durchführen darf.

\section{Modelle für Zugriffskontrolle}
\subsection{Discretionary Access Control (DAC)}
Subjekt kann Zugriffsrechte eines Objektes festlegen.

\subsection{Mandatory Access Control (MAC)}
Systemweite Regeln legen Zugriffsrechte fest.

\subsection{Role-based Access Control (RBAC)}
Subjekte werden Rollen zugewiesen. Rollen bestimmen Zugriffsrechte auf Objekte.

\subsection{Attribute-based Access Control (ABAC)}
Subjeke und Objekte haben Attribute, diese werden zur Zugriffskontrollentscheidung zu rate gezogen.

\section{Implementierungen}
In der Regel hybrider Ansatz aus mehreren Modellen.

\subsection{Unix}
Eingeschränktes ACL Modell mit Rechten für \qq{User}, \qq{Group} und \qq{Other}:
\begin{itemize}
    \item r: read
    \item w: write
    \item x: files: execute; dir: access (read file list)
\end{itemize}
Additionally:
\begin{itemize}
    \item t: directories: only file owner may delete
    \item s: setUID, setGID. File: execute with UID/GID; Dir: create files with UID/GID.
\end{itemize}

\subsection{Linux ACLs}
Linux unterstützt ACLs, es kann also für beliebige Objekte festgelegt werden, welche Subjekte welche Aktionen ausführen dürfen.

\section{Bell-LaPadula-Modell}
Definition von Zugriffsebenen. 
Klassifikation von Objekten nach Zugriffslevel (Security Level),
und Subjekten nach Freigabe (Clearance).

\vspace{.3cm}

Ziel: Informationsfluss zu niedrigereren Zugriffsleveln verhindern.

\vspace{.3cm}

Das System befindet sich zu jedem Zeitpunkt in einem Zustand.
Ein Zustand ist ein Tripel $v = (b, M, f)$ mit:
\begin{itemize}
    \item $b \subseteq S \times O \times A$:
        die Menge der aktuellen Zugriffe von Subjekten auf Objekte
    \item $M$ ist die aktuelle Zugriffsmatrix
    \item $f$ ist ein Tripel $(f_S, f_C, f_O)$ mit:
        \begin{itemize}
            \item $f_S: S \to SL$: Maximale Freigabe der Subjekte
            \item $f_C: S \to SL$: Aktuelle Freigabe der Subjekte
            \item $f_O: O \to SL$: Aktuelle Sicherheitsklassifikation der Objekte
        \end{itemize}
        Ein Subjekt kann sein $f_C$ frei in den Beschränkungen von $f_S$ wählen.
\end{itemize}

Ein Zustand ist sicher, falls gilt:
\begin{itemize}
    \item Simple-Security-Property: Für alle $(s, o, a) \in b$ mit $a=$ read oder $a=$ write gilt:
        $f_O(o) \leq f_S(s)$
    \item *-Eigenschaft:
        \begin{itemize}
            \item Für $a=$ append (kein lesen): $f_C(s) \leq f_O(o)$
            \item Für $a=$ write: $f_C(s) = f_O(o)$
            \item Für $a=$ read: $f_O(o) \leq f_C(s)$
        \end{itemize}
        Kurz: \qq{No read up} und \qq{No write down}.
    \item Discretionary Security Property: $\forall (s,o,a) \in b: a \in M[s,o]$
\end{itemize}

Basic Security Theorem: Wenn der Anfangszustand und alls Zustandsübergänge sicher sind, dann sind alle erreichbaren Folgezustände sicher.

\section{POSIX Capabilities}
Problem: Autorisierung kritischer Resources auf root beschränkt (z.B. Raw Sockets für Ping), workaround mit SUID-Binaries, Anwendung läuft dann aber faktisch mit root-Rechten, potentielle Sicherheitslücke.

\vspace{.3cm}
Lösung: POSIX Capabilitites, erlauben es einzelne Capabilitites freizuschalten.


\chapter{Grundlagen}
Zuverlässigkeit (Dependability) eines Systems besteht aus:
\begin{itemize}
    \item Verfügbarkeit (Availability)
    \item Verlässlichkeit (Reliability)
    \item Sicherheit (Safety)
    \item Integrität (Integrity)
    \item Wartbarkeit (Maintainability)
    \item Vertraulichkeit (Confidentiality)
\end{itemize}

Grundlegende Sicherheitsziele sind (Eselsbrücke CIA):
\begin{itemize}
    \item Vertraulichkeit (\textbf{C}onfidentiality):
        Schutz von Informationen vor unberechtigtem Zugriff
    \item Integrität (\textbf{I}ntegrity):
        Schutz vor unberechtigter Veränderung/Zerstörung
    \item Verfügbarkeit (\textbf{A}vailibility):
        Schutz vor Verhindern der Nutzbarkeit von Ressourcen
\end{itemize}
weitere Sicherheitsziele sind:
\begin{itemize}
    \item Identifikation bzw. Authentisierung:
        Zuordnung eines Identifikators bzw. Verifikation einer Identität
    \item Autorisierung bzw. Zugriffskontrolle:
        Zuordnung von Berechtigungen
    \item Verbindlichkeit:
        Garantierte Rückverfolgbarkeit von Aktionen zu Aktoren
    \item Datenschutz:
        Die Möglicheit selbst zu entscheiden welche Informationen preisgegeben werden sollen.
\end{itemize}

Sicherheitsziele werden durch \qq{Threats} bedroht. 
Diese sind Potentielle Fehler im System, die bei Ausnutzung Sicherheitsziele verletzen.

\vspace{0.3cm}

Schwachstellen (\qq{Vulnerabilities}) sind konkrete Schwachstellen, welche Sicherheitsziele bedrohen.

\vspace{0.3cm}

Diese werden von Angriffen (\qq{Attacks}) ausgenutzt.

\section{Kategoriesierung von Angriffen}
\subsection{Nach Intention}
\begin{itemize}
    \item Störung (DOS)
    \item Unerlaubter Informationszugriff
    \item Eindringen ins Netz/auf Knoten
    \item Veränderung von Information
\end{itemize}

\subsection{Nach Verfahren}
\begin{itemize}
    \item Verstellung (Masquerading) 
    \item Mithören (Eavesdropping)
    \item Zugriffsverletzung
    \item Verlust/Veränderung
    \item Verleugnung
    \item Fälschen
    \item Sabotage
\end{itemize}

\subsection{Nach Angriffspunkt}
\begin{itemize}
    \item Netzwerk
    \item Netzwerkdienste
    \item Betriebssystem/Anwendung
    \item Benutzer
\end{itemize}

\subsection{S.T.R.I.D.E. Kategorisierung (MS)}
\begin{itemize}
    \item Spoofing
    \item Tampering
    \item Repudiation
    \item Information disclosure
    \item Denial of Service
    \item Elevation of privilege
\end{itemize}

\section{Sicherheitsmechanismen und -richtlinien}
\subsection{Sicherheitsmechanismus}
Eine Methode, Werkzeug oder Vorgehen um eine Sicherheitsrichtlinie durchzusetzten.

Mögliche Mechanismen:
\begin{itemize}
    \item Verhinderung
    \item Erkennung
    \item Korrektur und Behebung
\end{itemize}

\subsection{Sicherheitsrichtlinie}
Ein Aussage was erlaubt, und was nicht erlaubt ist.

\section{Integration von Sicherheitsmechanismen}
\qq{Weakest link}-Konzept.

\section{Gesamtsystem-Sicht}
Risikoquellen:
\begin{itemize}
    \item Benutzer
    \item Organisatorische Probleme
    \item Spezifikation, Entwurf, Implementierung
\end{itemize}

Sicherheitsmaßnahmen:
\begin{itemize}
    \item Krypthographie
    \item Hardware-Bausteine
    \item Filter, Firewalls, Intrusion Detection
    \item Software Hardening gegen Fehler
    \item Organisatorische Regelung, Prozesse
    \item Gutes Design
\end{itemize}

Weitere Aspekte:
\begin{itemize}
    \item Kosten
    \item Gesetzliche Vorgaben
    \item Nutzerakzeptanz
\end{itemize}

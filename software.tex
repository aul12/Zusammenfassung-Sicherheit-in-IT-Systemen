\chapter{Malware und Software Security}
\section{Buffer Overflow}
\subsection{Stack-Buffer-Overflow}
Schreiben von ausführbarem Code in lokale Variable über Puffergrenze hinaus, Rücksprungadresse wird überschrieben und so der Code ausgeführt (oftmals Shellcode).

\vspace{.3cm}

Aufbau Stack (von hohen zu niedrigen Adressen):
\begin{enumerate}
    \item Argumente (ARG)
    \item Rücksprungadresse (RET)
    \item Basepointer (EBP)
    \item Lokale Variablen (VAR)
\end{enumerate}

\subsubsection{Lösungen}
\paragraph{Stack Canaries} 
Bekannte Werte zwischen Daten und EBP.

Mögliche Canary-Werte:
\begin{itemize}
    \item Terminator Canaries: Nutzen aus das Strings Nullterminiert sind. Nachteil: Canary Wert ist bekannt
    \item Random Canaries: Rein zufälliger Wert
    \item Random XOR Canaries: Zufälliger Wert $\oplus$ Kontrolldaten (z.B. Rücksprungadresse)
\end{itemize}

Vorteile:
\begin{itemize}
    \item Keine Systemweite Änderung
    \item Performanceeinflüsse nur bei entsprechend kompilierten Programmen
\end{itemize}
Nachteile: Programm (und alle Libraries) müssen mit Stack Canaries compiliert werden

\paragraph{XD und NX Bits}
Stack-Speicherseiten als nicht ausführbar markieren.

\paragraph{Typsichere Sprachen}
Keine Out-Of-Bounds zugriffe.

\paragraph{Address Space Layout Randomization (ASLR)}
Kein laden von Programmcode an determinisitischer Adresse möglich.

\paragraph{Code Signing}
Nur signierter Code wird ausgeführt.

\section{Begriffe}
\subsection{Error}
Design- oder Programmierfehler eines Entwicklers

\subsection{Fault/Bug}
Manifestation eines Error in einem Programm. Bugs können zu Sicherheitsproblemen führen.

\subsection{Failure}
Abweichung eines Programms vom spezifizierten Verhalten, soll durch Testing verhindert werden.

\section{Software Security}
Kritische sind:
\begin{itemize}
    \item Programme die erhöhte Privilegien haben
    \item Programme die über das Netwerk erreichbar sind
    \item Programme die Nutzereingaben irgendeiner Form akzeptieren:
        \begin{itemize}
            \item Direkte Benutzereingabe
            \item Kommandozeilenargumente
            \item Konfigurationsfiles
            \item Daten aus Netzwerkverbindungen
            \item Umgebungsvariablen
        \end{itemize}
\end{itemize}

\section{Schadsoftware}
\subsection{Unterscheidung anhand des Verbreitungswegs}
\begin{itemize}
    \item Virus: ein Programm, das sich verbreitet, indem es andere harmlose Programme benutzt
    \item Wurm: ein Programm, das sich selbstständig verbreitet
    \item Trojaner: ein Programm, das sich als harmloses Programm tarnt, aber eine schädliche Funktion beinhaltet
\end{itemize}

\subsection{Mögliche Schadroutinen}
\begin{itemize}
    \item Daten löschen
    \item Daten ausspionieren
    \item Fremdzugriff ermöglichen
    \item Rechner lahmlegen
    \item Computer zerstören
\end{itemize}

\subsection{Botnetze}
Ansammlung von infizierten PCs, die von zentralem Command \& Control (C\&C) Server kontrolliert werden 
und eine schädliche Funktion ausführen.

\vspace{.3cm}

Nutzung: Spam-Versand, Informationssammlung, DDOS, Clickfraud

\subsection{Targeted Attacks}
Angreifer verfügen über sehr detailliertes Wissen über das Zielsystem. Malware wird direkt im Zielsystems eingesetzt, z.B. mithilfe von Social Engineering.

\subsection{Ransomware}
Schadroutine: Sperrung des PCs, Verschlüsselung aller Daten,... bis Lösegeld bezahlt wurde.

\subsection{Weitere Aspekte}
\begin{itemize}
    \item Polymorphe Malware: verhindert Signaturbasierte Erkennung durch ständige Veränderung
    \item Antivirus Systeme: Signaturbasiert oder Verhaltensbasiert
    \item Antivirus Evasion: Erkennung ob in VM, ausführen nur während Bootprozess, Obfuscation
\end{itemize}

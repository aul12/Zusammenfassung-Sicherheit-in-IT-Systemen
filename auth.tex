\chapter{Identifikation und Authentisierung}
\section{Identität}
\begin{itemize}
    \item Eindeutig
    \item Unabänderlich
    \item Dauerhaft gültig
    \item Nicht übertragbar
\end{itemize}

\section{Identifikator}
Eine Menge von Merkmalen, die geeignet ist, die Identität einer Einheit zweifelsfrei festzustellen.

\vspace{.3cm}

Identifikatoren für Subjekte: Benutzer, Prozesse

Identifikatoren für Objekte: Filenamen, URIs

\vspace{.3cm}

Identifikatoren für Menschen: Filenamen, Mailadressen

Identifikatoren für Maschinen: Filehandles, Tabellen-Indizes

\vspace{.3cm}

Temporäre Identifikatoren: Aliase, UIDs

Persistente Identifikatoren: Passnummern

\subsection{Verwendung von Identifikatoren}
Zweck:
\begin{itemize}
    \item Zuordnenbarkeit (Accountability)
    \item Zugangskontrolle (Access-Control)
\end{itemize}

\section{Identifikation vs. Authentisierung}
Identifikation: Vorlage eines Identifikators (z.B. Personalausweis)

Authentisierung: Vorlage eines Identifikationsnachweises (z.B. Passwort)

Authentifizierung: Prüfung einer Authentisierung

\section{Authentisierung}
Drei Grundformen:
\begin{itemize}
    \item \qq{Something you know} (z.B. Password)
    \item \qq{Something you have} (z.B. Smartcard)
    \item \qq{Something you are} (z.B. Biometrie)
\end{itemize}

Häufig auch Kombination (2-Faktor-Authentisierung).

\section{Passwörter}
\subsection{Time Memory Trade-Off Angriffe}
Brute-force (online) Angriff vs. Code-book (offline) Angriff.

\subsection{Schutz gegen Rainbow Tables (Code-books)}
Ein zufälliger Salt Wert wird dem Password vor dem hashen zugefügt und mit dem Passwort gespeichert.

\subsection{Challenge-Response Verfahren}
Problem bei Passwörtern: Wiederverwendbarkeit.

Lösung: Einmalpasswörter

Ablauf:
\begin{enumerate}
    \item System $S$ und User $U$ vereinbaren eine geheime Funktion $f(c) = r$.
    \item $U$ identifiziert sich gegenüber $S$
    \item $S$ schickt $U$ eine Challenge $c$, $U$ antwortet mit $r$
    \item $S$ überprüft ob $r = f(c)$
\end{enumerate}

\subsubsection{S/Key-Ansatz}
System generiert Hash-Kette ausgehend von Seed $s$:
\begin{eqnarray*}
    p_1 &=& h(s) \\
    p_{n+1} &=& h(p_n)
\end{eqnarray*}

Bei der $i$-ten Authentisierung fragt $S$ nach dem Passwort $p_{n+1-i}$, die Hash-Kette wird also rückwärts durchlaufen.

\subsubsection{HMAC-based One-time Password Algorithm (HOTP)}
\begin{itemize}
    \item $K$: Geheimer Schlüssel
    \item $C$: Zähler (z.B. Zeitbasiert, bei Google: Unix-Time/30)
    \item Truncate$(\cdot)$: Auswahl von 4 bytes aus dem Resultat
    \item
        \begin{equation*}
            \text{HMAC}(K, C) = \text{SHA1}\left(K \oplus {5C5C\ldots}_\text{HEX} || 
                 \text{SHA1}\left(K \oplus {3636\ldots}_\text{HEX} || C\right)\right)
        \end{equation*}
    \item
        \begin{equation*}
            \text{HOTP}(K, C) = \text{Truncate}\left(\text{HMAC}(K, C)\right) \&
                {7FFFFFFF}_\text{HEX}
        \end{equation*}
    \item Mit $d$ als Anzahl der Ziffern:
        \begin{equation*}
            \text{HOTP-Value} = \text{HOTP}(K, C) \mod 10^d
        \end{equation*}
\end{itemize}

\section{Biometrische Authenisierung}
Biometrische Merkmale bieten keine Eindeutige Korrektheit, nur ein Grad für die Ähnlichkeit.

Tradeoff zwischen:
\begin{itemize}
    \item FAR: \qq{false acceptance rate}
    \item FRR: \qq{false rejection rate}
\end{itemize}
zusätzlich gibt es: EER (\qq{equal error rate}).

\section{Netzwerk Authentisierung}
Problem: Netzwerk kein vertrauenswürdiger Bereich: Klartext-Passworte unsicher, Nutzung von Challenge-Response oder kryptographischen Mechanismen.

\section{Zertifizierung}
Problem: wie wird sichergestellt, dass ein Schlüssel $S_x$ wirklich zu $X$ gehört?

Lösung: Vertrauenswürdige Verknüpfung von Schlüssel und Identität: Zertifikat.

Bestätigung durch \qq{Trusted Third Party} (TTP), z.B. CA.

Dafür: \qq{Public Key Infrastructure} (PKI).

\subsection{X.509 Zertifikate}
Standard, wird z.B. von SSL/HTTPS, S-MIME, LDAP, uvm. genutzt.

\subsection{PGP - Web of Trust}
Dezentraler Ansatz: Benutzer signieren die öffentlichen Schlüssel von aderen Benutzern.

Public-Keyring speichert öffentliche Schlüssel und Signaturen von eigenen und fremden PKs.

Owner Trust: Vertrauen in den Besitzer eines PK.

Calculated Trust: Vertrauen in die Korrektheit eines öffentlichen Schlüssels.

\subsection{Neuere Konzepte}
\begin{itemize}
    \item OAuth (Authentisierung bei Dienst $B$ ohne das Dienst $A$ Daten kennt)
    \item OpenID: Single Sign In
\end{itemize}

\chapter{Datenschutz \& Privacy}
\section{Begriffe}
\subsection{Privatssphäre / Privacy}
\begin{itemize}
    \item Kernbereich privater Lebensgestaltung
    \item Persönliche Unversertheit
    \item Identitätsmerkmale
    \item Persönlichkeitsentfaltung
\end{itemize}

\subsection{Datenschutz / Data protection}
Regelung und Abwehr von Datensammlung und Zugriff, welche die Privatsphäre beeinträchtigen würden.

\subsection{Gesetzliche Regelung}
Recht auf informationelle Selbstbestimmung.

\section{Datenschutzprinzipien}
\begin{itemize}
    \item Datenvermeidung: Keine Erhebung von Daten wo nicht notwendig
    \item Datensparsamkeit: Minimierung; nur Erhebung der zwingend notwendigen Daten
    \item Zweckbindung: Daten werden nur für den intendierten Zweck genutzt
    \item Informierte Zustimmung: Die Betroffene Persion ist über die Datenverarbeitung infromiert und stimmt ihr zu.
    \item Datenschutzbeauftragte: Betriebe, Behörden, Bundesländer, Bund haben Beauftragte die Einhaltung überwachen.
    \item Auskunftsrecht: Personen haben ein Recht auf Information in Bezug auf über sie erhobene Daten.
    \item Beschwerderecht: Bei Verstößen
    \item Korrekturrecht: Recht auf Korrektheit der Daten
    \item Recht auf Löschung oder Sperrung
    \item Widerspruchsrecht bei Weiterleitung
\end{itemize}

\section{Persönliche Daten}
\begin{itemize}
    \item Daten mit eindeutigem Personenbezug
    \item Daten die auf eine Person beziehbar sind
\end{itemize}

Begriffe:
\begin{itemize}
    \item Anonymität: Unidentifizierbarkeit
    \item Pseudonymität: Nutzung von Pseudonymen als Identifikator, kann von Anonym bis eindeutig Zuordnenbar sein
\end{itemize}

\section{Anonyme Kommunikation}
\subsection{Onion Routing}
\begin{itemize}
    \item Benutzer benötigt eine Liste von Kryptographischen Onion Routern (COR) public keys
    \item Benutzer definiert eine beliebige Route über mehrere Zwischenknoten
    \item Austausch individueller symmetrischer Schlüssel mit jedem Zwischenknoten
    \item Zwiebelschalenartige Verschlüssekung im Anonymisierungsnetz, Klartext am Exit Node.
\end{itemize}
z.B. Tor.

\vspace{.3cm}

Angriffsvektoren:
\begin{itemize}
    \item Packet correlation attacks
    \item Falsche Nutzung von DNS (sollte via Exit Router geschehen)
    \item Identifikation über Payload
\end{itemize}

\section{Blinde Signaturen}
Anforderungen:
\begin{itemize}
    \item Anonymität
    \item Verifizierbare Authentizität
    \item Keine Mehrfachverwendung
\end{itemize}

\section{Anonymous Credentials}
Beweis eines Attributes, ohne die eigene Identität preiszugeben. Außerdem keine Verknüpfbarkeit von zwei Authentisierungsvorgängen.

z.B.: Konzerttickets, Alternsnachweis, ÖPNV Ticket.

\section{Privacy in mobilen Netzen}
\subsection{GSM-UMTS}
Netzbetreiber können Engeräte lokalisieren.

\subsection{WLAN}
MAC-Adresse im Header jedes Pakets.

\section{Privacy im Web}
\subsection{Cookies}
\begin{itemize}
    \item HTTP Cookies (klassische Cookies)
    \item DOM Cookies (DOM Storage)
    \item Flash Cookies (local shared objects)
\end{itemize}
Oftmals auch: Browser Fingerprinting.

\section{Weitere Problemfelder}
\begin{itemize}
    \item Pay-as-you-drive Autoversicherungen
    \item Gesundheitskarten
\end{itemize}
